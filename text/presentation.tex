\documentclass[10pt, handout, aspectratio=169]{beamer}

\usepackage{amssymb,amsmath,mathtext}
\usepackage{indentfirst,amsfonts}
\usepackage{makecell,multirow,longtable}
\usepackage{graphicx}
\usepackage{color}
\usepackage{verbatim}
\usepackage{booktabs}
\usepackage{biblatex}
\addbibresource{biblio.bib}


\graphicspath{{graphs/}}

\usepackage[english,russian]{babel}
\usepackage[T2A]{fontenc}
\usepackage[utf8]{inputenc}

\setbeamertemplate{navigation symbols}{}
\setbeamertemplate{footline}[frame number]

\usetheme{pittsburgh}
\usecolortheme{seahorse}

\setbeamerfont{frametitle}{series=\bfseries}
\setbeamerfont{block title}{series=\bfseries}
\setbeamerfont{page number in head/foot}{size=\large}

\begin{document}
	\title{Нейросетевая реконструкция пористых тел с сохранением топологических и статистических свойств образца}
	\author{Будакян Я. С. \and \break \break \break Научный руководитель: к.т.н., доцент Грачев Е. А.}
	\maketitle
	
	\begin{frame}{Проблема исследований кернов}
		\begin{itemize}
			\item При моделировании геофизических процессов существует проблема недостаточности знаний о среде, в которой эти процессы протекают. Точные данные доступны из небольшой области (керна), а данные компьютерной томографии - из ещё меньшей области (кусочков, вырезанных из керна);
			\item Получение дополнительных данных (например, новых кернов, компьютерной томографии) связано с большими затратами. При этом, многие эксперименты в реальности с керном можно провести только один раз, поскольку они необратимым образом влияют на керн;
		\end{itemize}
	С решением задачи правдоподобного реконструирования пористых сред, появляется возможность:
		\begin{itemize}
			\item Провести апскейлинг (реконструировать среду в большем размере, чем оригинальный образец);
			\item Проводить статистические эксперименты (проводить моделирование процессов не только на оригинальном образце, но и на его реконструкциях, получая таким образом не значение в одной точке, а распределение);
		\end{itemize}
	\end{frame}

	\begin{frame}{Задачи}
		\begin{itemize}
			\item Разработать алгоритм реконструкции трехмерных пористых сред на основе образца, с сохранением следующих его топологических и статистических характеристик:
			\begin{itemize}
				\item 4 первых функционала Минковского:
				\begin{itemize}
					\item Объем
					\item Площадь поверхности
					\item Средняя кривизна
					\item Характеристика Эйлера-Пуанкаре
				\end{itemize}
				\item Двухточечная корреляционная функция
			\end{itemize}
			\item Верифицировать алгоритм на большом количестве реконструированных образцов
		\end{itemize}
	\end{frame}

	\begin{frame}{Функционалы Минковского}
		Функционалы Минковского для трехмерных тел вводятся следующим образом:
		\begin{itemize}
			\item $ \displaystyle V = M_0 = \int_{\delta X} dV $
			\item $ \displaystyle S = M_1 = \frac{1}{3} \int_{\delta X} dS $
			\item $ \displaystyle B = M_2 = \frac{1}{6} \int_{\delta X} \left ( \frac{1}{R_1} + \frac{1}{R_2} \right ) dS$
			\item $\displaystyle \xi = M_3 = \frac{1}{3} \int_{\delta X} \frac{1}{R_1 R_2} dS $
		\end{itemize}
	\end{frame}

	\begin{frame}{Метод решения}
		Для реконструкции применяется генеративная состязательная нейронная сеть
		\begin{figure}
			\centering{\includegraphics[width=0.7\linewidth]{GAN_overview}}
		\end{figure}
		Ее обучение сводится к минимизации функционала:
		\[ \underset{\theta}{\min} \underset{\zeta}{\max} \ \mathbb{E}_{x \sim p_{data}}\log D_\zeta(x) + \mathbb{E}_{z \sim p_{z}} \log (1 - D_\zeta(G_\theta(z))) \]
	\end{frame}

	\begin{frame}{Новизна}
		\begin{itemize}
			\item Конкретные архитектуры сетей генератора и дискриминатора взяты неизмененными из других работ похожей тематики;
			\item Новизна заключается в модифицированной адаптируемой процедуре обучения нейронной сети с автоматическим отслеживанием значений функционалов Минковского и изменением гиперпараметров в процессе обучения;
		\end{itemize}
	\end{frame}

	\begin{frame}{Текущие результаты}
		На данный момент:
		\begin{itemize}
			\item Разработка алгоритмической части завершена (реализован программный комплекс, позволяющий обучать нужны сети и использовать их для реконструкции образцов);
			\item Проведен ряд вычислительных экспериментов по обучению сетей на томографии песчаника;
			\item Анализ в процессе (анализ функционалов Минковского готов, двухточечная корреляционная функция пока нет);
			\item Текст готов примерно на треть;
		\end{itemize}
		Ещё нужно сделать:
		\begin{itemize}
			\item Провести дополнительные вычислительные эксперименты;
			\item Провести более широкомасштабный анализ функционалов Минковского реконструированных образцов;
			\item Провести статистический анализ (построить и сравнить двухточечные корреляционные функции);
		\end{itemize}
	\end{frame}

	\begin{frame}{Текущие результаты}
		\begin{columns}
			\column{0.25\linewidth}
			\begin{figure}
				\centering{\includegraphics[width=\linewidth]{original}}
				\caption{Образец}
			\end{figure}
			\column{0.25\linewidth}
			\begin{figure}
				\centering{\includegraphics[width=\linewidth]{generated_1}}
				\caption{1}
			\end{figure}
			\column{0.25\linewidth}
			\begin{figure}
				\centering{\includegraphics[width=\linewidth]{generated_2}}
				\caption{2}
			\end{figure}
			\column{0.25\linewidth}
			\begin{figure}
				\centering{\includegraphics[width=\linewidth]{generated_3}}
				\caption{3}
			\end{figure}
		\end{columns}
		\vfill
		\centering{Примеры реконструкции одной из обученных сетей}
	\end{frame}

	\begin{frame}{Текущие результаты}
		Анализ значений функционалов Минковского (V, S) для реконструированных образцов
		\begin{columns}
			\column{0.45\linewidth}
			\begin{figure}
				\centering{\includegraphics[width=\linewidth]{V_dispersion}}
			\end{figure}
			\column{0.45\linewidth}
			\begin{figure}
				\centering{\includegraphics[width=\linewidth]{S_dispersion}}
			\end{figure}
		\end{columns}
	\end{frame}

	\begin{frame}{Текущие результаты}
		Анализ значений функционалов Минковского (B, $\xi$) для реконструированных образцов
		\begin{columns}
			\column{0.45\linewidth}
			\begin{figure}
				\centering{\includegraphics[width=\linewidth]{B_dispersion}}
			\end{figure}
			\column{0.45\linewidth}
			\begin{figure}
				\centering{\includegraphics[width=\linewidth]{Xi_dispersion}}
			\end{figure}
		\end{columns}
	\end{frame}

\end{document}