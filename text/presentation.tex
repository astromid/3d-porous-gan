\documentclass[12pt]{beamer}

\usepackage{amssymb,amsmath,mathtext}
\usepackage{indentfirst,amsfonts}
\usepackage{makecell,multirow,longtable}
\usepackage{graphicx}
\usepackage{color}
\usepackage{verbatim}
\usepackage{booktabs}
\usepackage{biblatex}
\addbibresource{biblio.bib}


\graphicspath{{graphs/}}

\usepackage[english,russian]{babel}
\usepackage[T2A]{fontenc}
\usepackage[utf8]{inputenc}

\setbeamertemplate{navigation symbols}{}

\usetheme{boxes}
\usecolortheme{seahorse}

\setbeamerfont{frametitle}{series=\bfseries}
\setbeamerfont{block title}{series=\bfseries}

\begin{document}
	\title{Реконструкция трёхмерных пористых сред с использованием искусственных нейронных сетей}
	\author{Будакян Я. С. \and \break \break \break Научный руководитель: к.т.н., доцент Грачёв Е. А.}
	\date{Москва, 2019 г.} 

	\maketitle

	\begin{frame}{Введение}
		\begin{itemize}
			\item При моделировании геофизических процессов существует проблема недостаточности знаний о среде, в которой эти процессы протекают. Точные данные доступны из небольшой области (керна - небольшого куска породы, забранного из скважины), а данные компьютерной томографии - из ещё меньшей области (кусочков, вырезанных из керна);
			
			\item Получение дополнительных данных (например, новых кернов, компьютерной томографии) связано с большими затратами. При этом, многие эксперименты по установлению различных характеристик среды в реальности можно провести только один раз для одного керна, поскольку они необратимым образом влияют на него.
		\end{itemize}
	\end{frame}

	\begin{frame}{Задача реконструкции}
		Разработка алгоритма реконструкции синтетических образцов пористой среды на основе данных с реального образца. Новые образцы должны сохранять некоторые топологические и статистические свойства реального образца:
		\begin{itemize}
			\item $ \displaystyle V = \frac{1}{V_{all}} \int_{V_{pore}} dV = \frac{V_{pore}}{V_{all}}$ - пористость
			\item $ \displaystyle S = \frac{1}{V_{all}} \int_{\delta V_{pore}} dS $ - удельная площадь поверхности
		\end{itemize}
		
	\end{frame}

	\begin{frame}{Задача реконструкции}
		\begin{itemize}
			\item $ \displaystyle B = \frac{1}{V_{all}} \int_{\delta V_{pore}} \left ( \frac{1}{R_1} + \frac{1}{R_2} \right ) dS$ - удельная кривизна поверхности
			\item $\displaystyle \xi = \frac{1}{V_{all}} \int_{\delta V_{pore}} \frac{1}{R_1 R_2} dS $ - число Эйлера
			\item $ S_2(r) = \boldsymbol{P}(\boldsymbol{x} \in P, \boldsymbol{x}+\boldsymbol{r} \in P), \quad \boldsymbol{x}, \boldsymbol{r} \in \mathbb{R}^d $ - двухточечная функция вероятности,
		\end{itemize}
		где $\delta V_{pore}$ - граница двух фаз (среды и пор), $\boldsymbol{P}$ - это вероятность того, что две точки, отстоящие друг от друга на вектор $\boldsymbol{r}$, принадлежат одной фазе (обе являются порами).
	\end{frame}
	
	\begin{frame}{Модельные ограничения}
		Рассматриваются данные компьютерной томографии керна, состоящего из двух фаз - среды и пор, т. е. томограмма это бинарно-сегментированное трёхмерное изображение.
		\begin{figure}[h]
			\centering{\includegraphics[width=0.5\linewidth]{5-results/berea/original}}
			\vfill
			Образец томограммы керна
		\end{figure}
	\end{frame}
	
	\begin{frame}{Математическая формализация}
		Задачу реконструкции можно формализовать с помощью вероятностной постановки задачи обучения:
		\begin{itemize}
			\item Рассматривается многомерное пространство $X$, содержащее множество всех трёхмерных изображений $x$: $X = \{x\}$
			\item Есть обучающая выборка, состоящая из реальных томограмм $D = \{x_i\}, D \subset X$
			\item Считается, что  $D$ задаёт в $X$ вероятностное распределение $P_X : X \longrightarrow [0,1]$
		\end{itemize}
	\end{frame}
	
	\begin{frame}{Математическая формализация}
		Задача реконструкции трёхмерной пористой среды сводится к синтезу случайного изображения $x'$ из распределения, близкого к задаваемому обучающей выборкой:
		$$ P_{X'} \approx P_X, \quad x' \sim X'$$
		
		Для моделирования вероятностного распределения $P_X$ предлагается использовать генеративную состязательную нейронную сеть.
	\end{frame}
	
	\begin{frame}{GAN}
		Генеративные состязательные сети (GAN - Generative Adversarial Networks) были придуманы в 2014 году и достигли больших успехов в задачах моделирования сложных распределений.
		\begin{itemize}
			\item $ P_{X'} \approx P_X \Leftrightarrow \rho(P_{X'}, P_X) \longrightarrow \underset{P_{X'}}{\min} $
			\item $ X' = g_{\theta}(\cdot) \Rightarrow \rho(g_{\theta}(\cdot), P_X) \longrightarrow \underset{\theta}{\min}$
			\item В качестве $\rho$ можно использовать функцию потерь обученного классификатора
		\end{itemize}
	\end{frame}
	
	\begin{frame}{GAN}
		Используются две нейросети:
		\begin{itemize}
			\item $d_{\zeta}(x)$ - классификатор, \textbf{дискриминатор}
			\item $g_{\theta}(x)$ - сеть, трансформирующая входящий шум в элементы множества $X'$, \textbf{генератор}
		\end{itemize}
		Суть использования двух сетей состоит в том, что они обучаются совместно, конкурируя друг с другом.
		$$ \theta^* = \underset{\theta}{\arg\max} \left[ \underset{\zeta}{\min} L(\zeta, \theta) \right] $$
	\end{frame}
	
	\begin{frame}{GAN}
		Процесс обучения сети GAN принимает следующий вид:
		\begin{columns}
			\column{0.5\linewidth}
			\begin{itemize}
				\item Обучается дискриминатор при фиксированном генераторе
				\item Обучается генератор при фиксированном дискриминаторе
				\item Повторяется до сходимости параметров обеих моделей
			\end{itemize}
			\column{0.5\linewidth}
			\begin{figure}
				\centering{\includegraphics[width=\linewidth]{3-ann/gan-training}}
			\end{figure}
		\end{columns}
	\end{frame}

	\begin{frame}{Модификация}
		Использование GAN для реконструкции пористых сред уже исследовалось \footfullcite{Mosser2017}. Однако, главный недостаток предыдущих экспериментов состоит в ручном контроле процесса обучения сетей. 
		
		Целью данной работы было:
		\begin{itemize}
			\item Повторить ранее описанный работоспособный подход
			\item Провести модификацию процедуры обучения сетей для устранения необходимости ручного контроля
			\item Провести сравнительный анализ результатов с точки зрения сохранения топологических и статистических характеристик реконструированных образцов
		\end{itemize}
	\end{frame}

	\begin{frame}{Обучающая выборка}
		Обучающая выборка для сети была сформирована путём разрезания компьютерной томограммы песчаника размером $400^3$ вокселей на кубики размером $64^3$ вокселей с перекрытием в 16 вокселей.
		\begin{table}
			\centering
			\begin{tabular}{p{3cm} p{3cm} p{3cm}}
				\includegraphics[width=1.0\linewidth]{5-results/berea/berea_1}
				&
				\includegraphics[width=1.0\linewidth]{5-results/berea/berea_3}
				&
				\includegraphics[width=1.0\linewidth]{5-results/berea/berea_5}
			\end{tabular}
			\caption{Примеры из сформированной обучающей выборки}
		\end{table}
	\end{frame}

	\begin{frame}{Результаты}
		Были проведены вычислительные эксперименты по обучению сетей с модификацией для устранения ручного контроля. Анализ реконструкций был произведён на размерах $64^3$, $216^3$ и $360^3$. Для каждого из размеров было получено >500 реконструкций, что позволило построить распределения их характеристик.
	\end{frame}
	
	\begin{frame}{Реконструкции $64^3$}
		\begin{table}
			\centering
			\begin{tabular}{p{3cm} p{3cm} p{3cm}}
				\includegraphics[width=1\linewidth]{5-results/analysis_64/generated/1}
				&
				\includegraphics[width=1\linewidth]{5-results/analysis_64/generated/3}
				&
				\includegraphics[width=1\linewidth]{5-results/analysis_64/generated/5}
				\\
				\includegraphics[width=1\linewidth]{5-results/analysis_64/generated/7}
				&
				\includegraphics[width=1\linewidth]{5-results/analysis_64/generated/9}
				&
				\includegraphics[width=1\linewidth]{5-results/analysis_64/generated/11}
			\end{tabular}
			\caption{Примеры реконструкций размера $64^3$}
		\end{table}
	\end{frame}
	
	\begin{frame}{Реконструкции $64^3$}
		\begin{figure}
			\begin{minipage}{0.45\linewidth}
				\centering
				\includegraphics[width=1\linewidth]{5-results/analysis_64/V_exp}
			\end{minipage}
			\hfill
			\begin{minipage}[h]{0.45\linewidth}
				\centering
				\includegraphics[width=1\linewidth]{5-results/analysis_64/S_exp}
			\end{minipage}
			\vfill
			\begin{minipage}[h]{0.45\linewidth}
				\centering
				\includegraphics[width=1\linewidth]{5-results/analysis_64/B_exp}
			\end{minipage}
			\hfill
			\begin{minipage}[h]{0.45\linewidth}
				\centering
				\includegraphics[width=1\linewidth]{5-results/analysis_64/Xi_exp}
			\end{minipage}
		\end{figure}
	\end{frame}

	\begin{frame}{Реконструкции $216^3$}
		\begin{table}
			\centering
			\begin{tabular}{p{3cm} p{3cm} p{3cm}}
				\includegraphics[width=1\linewidth]{5-results/analysis_216/generated/1}
				&
				\includegraphics[width=1\linewidth]{5-results/analysis_216/generated/3}
				&
				\includegraphics[width=1\linewidth]{5-results/analysis_216/generated/5}
				\\
				\includegraphics[width=1\linewidth]{5-results/analysis_216/generated/7}
				&
				\includegraphics[width=1\linewidth]{5-results/analysis_216/generated/9}
				&
				\includegraphics[width=1\linewidth]{5-results/analysis_216/generated/11}
			\end{tabular}
			\caption{Примеры реконструкций размера $216^3$}
		\end{table}
	\end{frame}

	\begin{frame}{Реконструкции $216^3$}
		\begin{figure}
			\begin{minipage}{0.45\linewidth}
				\centering
				\includegraphics[width=1\linewidth]{5-results/analysis_216/V_exp}
			\end{minipage}
			\hfill
			\begin{minipage}[h]{0.45\linewidth}
				\centering
				\includegraphics[width=1\linewidth]{5-results/analysis_216/S_exp}
			\end{minipage}
			\vfill
			\begin{minipage}[h]{0.45\linewidth}
				\centering
				\includegraphics[width=1\linewidth]{5-results/analysis_216/B_exp}
			\end{minipage}
			\hfill
			\begin{minipage}[h]{0.45\linewidth}
				\centering
				\includegraphics[width=1\linewidth]{5-results/analysis_216/Xi_exp}
			\end{minipage}
		\end{figure}
	\end{frame}

	\begin{frame}{Реконструкции $216^3$, ручной контроль}
		\begin{figure}
			\begin{minipage}{0.45\linewidth}
				\centering
				\includegraphics[width=1\linewidth]{5-results/analysis_216/V_paper}
			\end{minipage}
			\hfill
			\begin{minipage}[h]{0.45\linewidth}
				\centering
				\includegraphics[width=1\linewidth]{5-results/analysis_216/S_paper}
			\end{minipage}
			\vfill
			\begin{minipage}[h]{0.45\linewidth}
				\centering
				\includegraphics[width=1\linewidth]{5-results/analysis_216/B_paper}
			\end{minipage}
			\hfill
			\begin{minipage}[h]{0.45\linewidth}
				\centering
				\includegraphics[width=1\linewidth]{5-results/analysis_216/Xi_paper}
			\end{minipage}
		\end{figure}
	\end{frame}
	
	\begin{frame}{Заключение}
		\begin{itemize}
			\item Воспроизведён подход по реконструкции пористой среды с помощью GAN
			\item Реализована модификация процесса обучения сети для устранения ручного контроля
			\item Получены результаты реконструкций для разных размеров
			\item Проведён сравнительный анализ характеристик реконструкций
		\end{itemize}
		Полученные результаты показывают, что сеть без ручного контроля обучения успешно обучается и способна реконструировать пористую среду, однако качество реконструкций на размерах больших, чем размер обучающих примеров, получилось хуже, чем для сети с ручным контролем обучения.
	\end{frame}
	
	\begin{frame}
		\centering\huge{Спасибо за внимание!}
	\end{frame}
	
	\begin{frame}{Задача минимизации}
		Обучение нейронной сети является задачей многопараметрической минимизации функционала потерь. Для используемых в этой работе сетей данная задача ставится так:
		 $$\mathcal{L}(\theta, \zeta) =  \mathbb{E}_{x \sim p_{data}}\log D_\zeta(x) + \mathbb{E}_{z \sim p_{z}} \log (1 - D_\zeta(G_\theta(z))) $$
		 $$\theta^*, \zeta^* = \underset{\theta}{\arg\min} \ \underset{\zeta}{\arg\max} \ \mathcal{L}(\theta, \zeta)$$
	\end{frame}
	
	\begin{frame}{Архитектуры G и D}
		\begin{table}[h]
			\tabcolsep=0.11cm
			\scriptsize
			\centering
			\begin{tabular}{|c|c|c|c|}
				\hline
				Слой & Размер ядра & Размерность выхода & Кол-во параметров \\
				\hline
				0\_ConvTranspose3d &  [256, 512, 4, 4, 4] & [1, 256, 4, 4, 4] & 8 388 610 \\
				1\_BatchNorm3d &  [256] & [1, 256, 4, 4, 4] & 512 \\
				2\_ReLU & - & [1, 256, 4, 4, 4] & - \\
				\hline
				3\_ConvTranspose3d &  [128, 256, 4, 4, 4] & [1, 128, 8, 8, 8] & 2 097 150 \\
				4\_BatchNorm3d &  [128] & [1, 128, 8, 8, 8] & 256 \\
				5\_ReLU & - & [1, 128, 8, 8, 8] & - \\
				\hline
				6\_ConvTranspose3d &  [64, 128, 4, 4, 4] & [1, 64, 16, 16, 16] & 524 290 \\
				7\_BatchNorm3d &  [64] & [1, 64, 16, 16, 16] & 128 \\
				8\_ReLU & - & [1, 64, 16, 16, 16] & - \\
				\hline
				9\_ConvTranspose3d &  [32, 64, 4, 4, 4] & [1, 32, 32, 32, 32] & 131 070 \\
				10\_BatchNorm3d &  [32] & [1, 32, 32, 32, 32] & 64 \\
				11\_ReLU & - & [1, 32, 32, 32, 32] & - \\
				\hline
				12\_ConvTranspose3d &  [1, 32, 4, 4, 4] & [1, 1, 64, 64, 64] & 2 050 \\
				13\_Tanh & - & [1, 1, 64, 64, 64] & - \\
				\hline
			\end{tabular}
			\caption{Архитектура генератора}
			\label{5-g-arch}
		\end{table}
	\end{frame}
	
	\begin{frame}{Архитектуры G и D}
		\begin{table}[h]
			\tabcolsep=0.11cm
			\scriptsize
			\centering
			\begin{tabular}{|c|c|c|c|}
				\hline
				Слой & Размер ядра & Размерность выхода & Кол-во параметров \\
				\hline
				0\_Conv3d &  [1, 32, 4, 4, 4] & [1, 32, 32, 32, 32] & 2 050 \\
				1\_LeakyReLU & - & [1, 32, 32, 32, 32] & - \\
				\hline
				2\_Conv3d &  [32, 64, 4, 4, 4] & [1, 64, 16, 16, 16] & 131 070 \\
				3\_BatchNorm3d &  [64] & [1, 64, 16, 16, 16] & 128 \\
				4\_LeakyReLU & - & [1, 64, 16, 16, 16] & - \\
				\hline
				5\_Conv3d &  [64, 128, 4, 4, 4] & [1, 128, 8, 8, 8] & 524 290 \\
				6\_BatchNorm3d &  [128] & [1, 128, 8, 8, 8] & 256 \\
				7\_LeakyReLU & - & [1, 64, 16, 16, 16] & - \\
				\hline
				8\_Conv3d &  [128, 256, 4, 4, 4] & [1, 256, 4, 4, 4] & 2 097 150 \\
				9\_BatchNorm3d &  [256] & [1, 256, 4, 4, 4] & 512 \\
				10\_LeakyReLU & - & [1, 256, 4, 4, 4] & - \\
				\hline
				11\_Conv3d &  [256, 1, 4, 4, 4] & [1, 1, 1, 1, 1] & 16 380 \\
				12\_Sigmoid & - & [1, 1, 1, 1, 1] & - \\
				\hline
			\end{tabular}
			\caption{Архитектура дискриминатора}
			\label{5-d-arch}
		\end{table}
	\end{frame}

\end{document}