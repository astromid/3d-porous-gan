\clearpage
\section{Верификация}

	Для того, чтобы оценить ``правдоподобие'' реконструкций, необходима процедура верификации. В данной работе верификация заключается в проверке сохранения топологических и статистических свойств реконструированных образцов по сравнению с обучающими - четырёх функционалов Минковского и двухточечной корреляционной функции.
	
	\subsection{Функционалы Минковского}
		Функционалы Минковского для трёхмерных изображений в общем случае вводятся следующим образом:
		\begin{itemize}
			\item $ \displaystyle M_0 = \int_{X} dV $
			\item $ \displaystyle M_1 = \frac{1}{3} \int_{\delta X} dS $
			\item $ \displaystyle M_2 = \frac{1}{6} \int_{\delta X} \left ( \frac{1}{R_1} + \frac{1}{R_2} \right ) dS$
			\item $\displaystyle M_3 = \frac{1}{3} \int_{\delta X} \frac{1}{R_1 R_2} dS $
		\end{itemize}
		Реконструкции являются бинарными трёхмерными изображениями, содержащими две фазы - среду и поры (пустоту). В дальнейшем все расчёты функционалов приведены для фазы пор. Также, поскольку абсолютные значения функционалов зависят от размера реконструкции, в дальнейшем рассматриваются их значения, нормированные на общий объем реконструкции. В этом случае, определения интересующих характеристик можно переписать следующим образом:
		\begin{itemize}
			\item $ \displaystyle V = \frac{1}{V_{all}} \int_{V_{pore}} dV = \frac{V_{pore}}{V_{all}}$
			\item $ \displaystyle S = \frac{1}{V_{all}} \int_{\delta V_{pore}} dS $
			\item $ \displaystyle B = \frac{1}{V_{all}} \int_{\delta V_{pore}} \left ( \frac{1}{R_1} + \frac{1}{R_2} \right ) dS$
			\item $\displaystyle \xi = \frac{1}{V_{all}} \int_{\delta V_{pore}} \frac{1}{R_1 R_2} dS, $
		\end{itemize}
		где $\delta V_{pore}$ - граница двух фаз (среды и пор).
		С точки зрения интерпретации, характеристики $ V, S, B, \xi $ соответствуют пористости, удельной площади поверхности, удельной кривизны поверхности и числу Эйлера-Пуанкаре (некоторая характеристика связности).
		
		Поскольку реконструкции представляют собой дискретизированные бинарные изображения, для расчёта функционалов можно не численно вычислять интегралы, а воспользоваться эффективным алгоритмом для дискретного случая \cite{Blasquez}.
	
	\subsection{Двухточечная функция вероятности}
		
		Статистические характеристики пористой среды можно выразить с помощью двухточечной функции вероятности. Предполагая стационарность, эта функция совпадает с нецентрированной ковариацией \cite{Matheron, Mosser}:
		$$ S_2(r) = \boldsymbol{P}(\boldsymbol{x} \in P, \boldsymbol{x}+\boldsymbol{r} \in P), \quad \boldsymbol{x}, \boldsymbol{r} \in \mathbb{R}^d $$
		$\boldsymbol{P}$ - это вероятность того, что две точки, отстоящие друг от друга на вектор $\boldsymbol{r}$, принадлежат одной фазе (обе являются порами). Для расчёта радиальной функции вероятности сначала рассчитываются соответствующие функции вдоль $x, y, z$, а потом усредняются. 