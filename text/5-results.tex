\clearpage
\section{Результаты вычислительных экспериментов}
	Архитектуры, описанные в пункте 1.4.3, а так же модификация процедуры обучения, описанная в 1.4.4 были реализованы в виде комплекса программ на языке Python с помощью библиотеки для глубокого обучения PyTorch. Использованные версии программных пакетов указаны в Приложении 1. Обучение проводилось на данных компьютерной томографии песчаника Berea.
	\subsection{Набор данных Berea}
		TODO
		
		\begin{figure}[h!]
			\centering{\includegraphics[width=0.5\linewidth]{5-results/berea/original}}
			\caption{Оригинальный образец}
			\label{5}
		\end{figure}
	
	
		\begin{table}[h!]
			\begin{center}
				\begin{tabular}{p{5cm} p{5cm} p{5cm}}
					\toprule
					\includegraphics[width=1\linewidth]{5-results/berea/berea_1}
					&
					\includegraphics[width=1\linewidth]{5-results/berea/berea_2}
					&
					\includegraphics[width=1\linewidth]{5-results/berea/berea_3}
					\\
					\includegraphics[width=1\linewidth]{5-results/berea/berea_4}
					&
					\includegraphics[width=1\linewidth]{5-results/berea/berea_5}
					&
					\includegraphics[width=1\linewidth]{5-results/berea/berea_6}
					\\
					\hline
				\end{tabular}
				\caption{Примеры из обучающей выборки}
				\label{5-berea64}
			\end{center}
		\end{table}
	\subsection{Описание эксперимента}
		Отличия в параметрах обучения нейросетей указаны в (Таб. \ref{todo}).
		
		\begin{table}[h!]
			\begin{center}
				\begin{tabular}{|c|c|c|c|}
					\hline
					Эксперимент & Число фильтров G & Число фильтров D & Политика LR \\
					\hline
					1 (25/02) & 64 & 64 & Constant \\
					\hline
					2 (27/03) & 64 & 64 & Constant \\
					\hline
					3 (13/04) & 64 & 64 & LR scheduler \\
					\hline
					4 (16/04) & 64 & 16 & LR scheduler \\
					\hline
					5 (18/04) & 64 & 64 & LR scheduler \\
					\hline
					6 (29/04) & 64 & 32 & LR scheduler \\
					\hline
				\end{tabular}
				\caption{Отличия в параметрах обучения моделей}
				\label{todo}
			\end{center}
		\end{table}
	
		\subsubsection{Примеры синтеза}
	
			Примеры синтеза, полученные с помощью нейросетей с различными параметрами, приведены в (Таб. \ref{8-dataset1-images}).
			
			\begin{table}[h!]
				\begin{center}
					\begin{tabular}{p{5cm} p{5cm} p{5cm}}
						\toprule
						% X & X & X & X \\
						% \cmidrule(r){1-1}\cmidrule(lr){2-2}\cmidrule(lr){3-3}\cmidrule(lr){4-4}
						\includegraphics[width=1\linewidth]{5-results/berea/original}
						&
						\includegraphics[width=1\linewidth]{5-results/berea/original}
						&
						\includegraphics[width=1\linewidth]{5-results/berea/original}
						\\
						\includegraphics[width=1\linewidth]{5-results/berea/original}
						&
						\includegraphics[width=1\linewidth]{5-results/berea/original}
						&
						\includegraphics[width=1\linewidth]{5-results/berea/original}
						\\
						\includegraphics[width=1\linewidth]{5-results/berea/original}
						&
						\includegraphics[width=1\linewidth]{5-results/berea/original}
						&
						\includegraphics[width=1\linewidth]{5-results/berea/original}
						\\
						\hline
					\end{tabular}
					\caption{Примеры реконструкции 64x64x64}
					\label{8-dataset1-images}
				\end{center}
			\end{table}
			
		\subsubsection{Топологические характеристики}
		
			Графики функционалов Минковского.
			
			\begin{figure}[h!]
				\centering{\includegraphics[width=0.5\linewidth]{5-results/exp_02_25/V}}
				\caption{V}
				\label{V}
			\end{figure}
		
			\begin{figure}[h!]
				\centering{\includegraphics[width=0.5\linewidth]{5-results/exp_02_25/S}}
				\caption{S}
				\label{S}
			\end{figure}
	
			\begin{figure}[h!]
				\centering{\includegraphics[width=0.5\linewidth]{5-results/exp_02_25/B}}
				\caption{B}
				\label{B}
			\end{figure}
			
			\begin{figure}[h!]
				\centering{\includegraphics[width=0.5\linewidth]{5-results/exp_02_25/Xi}}
				\caption{Xi}
				\label{Xi}
			\end{figure}
		
		\subsubsection{Статистические характеристики}
			
			Графики радиальной функции вероятности и двухточечной корреляционной функций:
			
			\begin{figure}[h!]
				\centering{\includegraphics[width=0.5\linewidth]{5-results/exp_02_25/radial_avg}}
				\caption{radial avg}
				\label{radial_avg}
			\end{figure}
			
			\begin{figure}[h!]
				\centering{\includegraphics[width=0.5\linewidth]{5-results/exp_02_25/corr_func}}
				\caption{corr func}
				\label{corr_func}
			\end{figure}
		
		\subsubsection{Анализ}
