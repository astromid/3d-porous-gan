\clearpage
\section*{\hfil ВЫВОДЫ \hfil}
\addcontentsline{toc}{section}{ВЫВОДЫ}
	Графики сходимости функционалов во время обучения (Рис. \ref{V-plot}, \ref{S-plot}, \ref{B-plot}, \ref{Xi-plot}) показывают, что на размере реконструкций $64^3$, на которых и обучается модель, они хорошо сходятся и остаются в зелёной зоне (разброс значений функционалов в обучающей выборке). Хорошее качество реконструкций $64^3$ подтверждает и анализ распределений выборки реконструированных образцов (Рис. \ref{5-dist-V-64}, \ref{5-dist-S-64}, \ref{5-dist-B-64}, \ref{5-dist-Xi-64}, \ref{5-prob-64}). Это даёт уверенность в том, что модели вообще обучаемы в автоматическом режиме, без ручного контроля шага обучения.
	
	Анализ реконструкций размеров $216^3$ и $360^3$ показывает менее впечатляющие результаты. Видно, что сеть, обученная с модификацией процедуры обучения теряет обобщающую способность - распределения функционалов оказываются смещены относительно оригинальных распределений, в то время как сеть \cite{Mosser} всё ещё показывает хорошее совпадение. Причём, сравнивая графики для $216^3$ с графиками для $360^3$ видно, что чем больше размер реконструкции выходит за размер обучаемых образцов, тем сильнее наблюдается смещение распределений.
	