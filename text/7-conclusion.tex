\clearpage
\section*{\hfil ЗАКЛЮЧЕНИЕ \hfil}
\addcontentsline{toc}{section}{ЗАКЛЮЧЕНИЕ}
	Была разработана модификация процедуры обучения нейронных сетей для реконструкции трёхмерных пористых сред, реализован комплекс программ и проведены вычислительные эксперименты. По результатам экспериментов с помощью обученных в их ходе сетей были реконструирована выборка образцов для проведения верификации с помощью сравнения статистических и топологических характеристик выборки с характеристиками изначального образца. Также было проведено сравнение с предыдущей работой в этой области \cite{Mosser}. По результатам экспериментов было выявлено, что внесённая в процесс обучения модификация не нарушает ход обучения (реконструкции размера, близкого к обучающему показывают хорошее качество), однако задачу апскейлинга сеть начинает решать хуже, проигрывая в обобщающей способности сети из работы \cite{Mosser}, которая была обучения при ручном контроле параметра шага обучения.
	
	В будущих работах возможно проведение дополнительных исследований на тему улучшения качества работы предложенного подхода в задаче апскейлинга. Также, интересной темой для развития исследования является попытка использования новых достижений в области генеративных состязательных сетей в рамках задачи реконструкции пористых сред.